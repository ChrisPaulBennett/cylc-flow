\subsection{Comparison With Existing Schedulers} 
\label{ComparisonWithExistingSchedulers}

As explained earlier, as far as the author is aware current forecast
schedulers ignore intercycle dependencies and therefore require
strict sequential cycling, which is very much sub-optimal outside of
normal real time operation. The simplest of these schedulers places the
burden of scheduling almost entirely on the user, who must supply a list
of tasks that the system will run through in order, perhaps with some
way of manually specifying crude functional parallelism.  This is
clearly sub-optimal even for quite simple systems and strict sequential
cycling, and task ordering has to be re-evaluated manually whenever the
system is changed. 

Another approach is hardwired system-specific finite state scheduling
logic that enforces a predetermined order of events: {\em if tasks A and
B have finished, then start task C} and so on. Optimal scheduling within
a forecast cycle is possible in this case, assuming correct coding, but
the hard coded logic inevitably becomes convoluted and inflexible as
system complexity increases, and handling to intercycle dependencies is
almost certainly not feasible.  

Finally, the well established SMS system from ECMWF is an industrial
strength general scheduling tool for dependent jobs. Users define ``SMS
suites'' that group tasks into ``families'' and define explicit
dependency relationships between tasks and/or/? families. SMS does not
appear to be able to do optimal cycle-independent scheduling, however,
or at least is not able to do so when configured for normal usage (THIS
IS STILL NOT ENTIRELY RESOLVED!) because (a) whole families trigger 
at once(?), rather than individual tasks, and global cycling mechanisms
are used to advance the system forward in time (either looping
over successive analysis times for ``catch up'' OR triggering off the
wall clock for real time operation. Thus, in contrast to cylc, it would
not be possible to have a delayed parallel test system catch up to the
main operation and then keep pace with it.


