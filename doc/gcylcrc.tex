
\section{Gcylc Config File Reference}
\label{GcylcRCReference}

\lstset{language=bash}

This section defines all legal items and values for the gcylc user
config file, which should be located in
\lstinline=$HOME/.cylc/gcylc.rc=.
Current settings can be printed with the \lstinline=cylc get-gui-config=
command.

\subsection{Top Level Items}

\subsubsection{initial views}

Set the suite view panel(s) displayed initially, when the GUI starts.
This can be changed later using the tool bar.

\begin{myitemize}
\item {\em type:} string (a list of one or two view names)
\item {\em legal values:} ``text'', ``dot'',  ``graph''
\item {\em default:} ``text''
\item {\em example:} \lstinline@initial views = graph, dot@
\end{myitemize}

\subsubsection{ungrouped views}

List suite views, if any, that should be displayed initially in an
ungrouped state. Namespace family grouping can be changed later
using the tool bar.

\begin{myitemize}
\item {\em type:} string (a list of zero or more view names)
\item {\em legal values:} ``text'', ``dot'',  ``graph''
\item {\em default:} (none)
\item {\em example:} \lstinline@ungrouped views = text, dot@
\end{myitemize}

\subsubsection{use theme}

Set the task state color theme, common to all views, to use
initially. The color theme can be changed later using the tool bar. 
See \lstinline@gcylc.rc.eg@ and \lstinline@themes.rc@ in
\lstinline@$CYLC_DIR/conf/gcylcrc/@ for how to modify existing color
themes or define your own. Use \lstinline@cylc get-gui-config@
to list your available themes.

\begin{myitemize}
\item {\em type:} string (theme name)
\item {\em legal values:} ``default'', ``solid'', ``high-contrast'',
    ``color-blind'', and any custom or user-modified themes.
\item {\em default:} ``default''
\end{myitemize}

\subsubsection{initial side-by-side views}

Set the suite view panels initial orientation when the GUI starts.
This can be changed later using the "View" menu "Toggle views side-by-side"
 option.

\begin{myitemize}
\item {\em type:} boolean (False or True)
\item {\em legal values:} ``False'', ``True''.
\item {\em default:} ``False''
\end{myitemize}

\subsubsection{task states to filter out}

Set the initial filtering options when the GUI starts. Later this can be
changed by using the "View" menu "Task Filtering" option.

\begin{myitemize}
\item {\em type:} string list
\item {\em legal values:} waiting, held, queued, ready, expired, submitted,
submit-failed, submit-retrying, running, succeeded, failed, retrying, runahead
\item {\em default:} runahead
\end{myitemize}

\subsubsection{dot icon size}

Set the size of the task state dot icons displayed in the text and dot
views.

\begin{myitemize}
\item {\em type:} string
\item {\em legal values:} ``small'' (10px), ``medium'' (14px), ``large'' (20px)
\item {\em default:} ``medium''
\end{myitemize}

\subsubsection{sort by definition order}

If this is not turned off the default sort order for task names and
families in the dot and text views will the order they appear in the
suite definition. Clicking on the task name column in the treeview will
toggle to alphanumeric sort, and a View menu item does the same for the
dot view.  If turned off, the default sort order is alphanumeric and
definition order is not available at all.

\begin{myitemize}
\item {\em type:} boolean
\item {\em default:} True
\end{myitemize}

\subsubsection{task filter highlight color}

The color used to highlight active task filters in gcylc. It must be a name
from the X11 rgb.txt file, e.g.\ \lstinline=SteelBlue=; or a
{\em quoted} hexadecimal color code, e.g.\ \lstinline="#ff0000"= for red (quotes
are required to prevent the hex code being interpreted as a comment).

\begin{myitemize}
    \item {\em type:} string
    \item {\em default:} \lstinline=PowderBlue=
\end{myitemize}

\subsubsection{window size}

Sets the size of the cylc gui at startup.

\begin{myitemize}
    \item {\em type:} integer list (x, y)
\item {\em legal values:} positive integers
\item {\em default:} 800, 500
\item {\em example:} \lstinline@window size = 1000, 700@
\end{myitemize}

\subsubsection{sort column}

If ``text'' is in \lstinline@initial views@ then \lstinline@sort column@ sets
the column that will be sorted initially when the gui launches. Sorting can be
changed later by clicking on the column headers.

\begin{myitemize}
    \item {\em type:} string
    \item {\em legal values:} ``task'', ``state'', ``host'', ``job system'',
        ``job ID'', ``T-submit'', ``T-start'', ``T-finish'', ``dT-mean'',
        ``latest message'', ``none''
    \item {\em default:} ``none''
    \item {\em example:} \lstinline@sort column = T-start@
\end{myitemize}

\subsubsection{sort column ascending}

For use in combination with \lstinline@sort column@, sets weather the column will
be sorted using ascending or descending order.

\begin{myitemize}
    \item {\em type:} boolean
    \item {\em legal values:} ``True'' (ascending), ``False'' (descending)
    \item {\em default:} ``True''
    \item {\em example:} \lstinline@sort column ascending = False@
\end{myitemize}

\subsection{[themes]}

This section may contain task state color theme definitions.

\subsubsection[{[}THEME{]}]{[themes] \textrightarrow [[THEME]]}

The name of the task state color-theme to be defined in this section.

\begin{myitemize}
\item {\em type:} string
\end{myitemize}

\paragraph[inherit]{[themes] \textrightarrow [[THEME]] \textrightarrow inherit}

You can inherit from another theme in order to avoid defining all states.

\begin{myitemize}
\item {\em type:} string (parent theme name)
\item {\em default:} ``default''
\end{myitemize}

\paragraph[defaults]{[themes] \textrightarrow [[THEME]] \textrightarrow defaults}

Set default icon attributes for all state icons in this theme.

\begin{myitemize}
\item {\em type:} string list (icon attributes)
\item {\em legal values:} \lstinline@"color=COLOR"@, \lstinline@"style=STYLE"@, \lstinline@"fontcolor=FONTCOLOR"@
\item {\em default:} (none)
\end{myitemize}

For the attribute values, COLOR and FONTCOLOR can be color names from the X11
rgb.txt file, e.g.\ \lstinline=SteelBlue=; or hexadecimal color codes, e.g.
\lstinline@#ff0000@ for red; and STYLE can be ``filled'' or ``unfilled''. 
See \lstinline@gcylc.rc.eg@ and \lstinline@themes.rc@ in
\lstinline@$CYLC_DIR/conf/gcylcrc/@ for examples. 

\paragraph[STATE]{[themes] \textrightarrow [[THEME]] \textrightarrow STATE}

Set icon attributes for all task states in THEME, or for a subset of them if 
you have used theme inheritance and/or defaults. Legal values of STATE are
any of the cylc task proxy states: {\em waiting, runahead, held, queued, ready, 
submitted, submit-failed, running, succeeded, failed, retrying, submit-retrying}.
 
\begin{myitemize}
\item {\em type:} string list (icon attributes)
\item {\em legal values:} \lstinline@"color=COLOR"@, \lstinline@"style=STYLE"@, \lstinline@"fontcolor=FONTCOLOR"@
\item {\em default:} (none)
\end{myitemize}

For the attribute values, COLOR and FONTCOLOR can be color names from the X11
rgb.txt file, e.g.\ \lstinline=SteelBlue=; or hexadecimal color codes, e.g.
\lstinline@#ff0000@ for red; and STYLE can be ``filled'' or ``unfilled''. 
See \lstinline@gcylc.rc.eg@ and \lstinline@themes.rc@ in
\lstinline@$CYLC_DIR/conf/gcylcrc/@ for examples. 
