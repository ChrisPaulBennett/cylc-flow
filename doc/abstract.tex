\begin{abstract}

    {\em Cylc} (pronounced ``silk'') is an advanced
    metascheduler\footnote{A metascheduler determines when dependent
    jobs are {\em ready} to run, at which point they can be sent to a
    batch queue scheduler. We drop the ``meta'' prefix from here on,
    however, because a metascheduler is also a type of scheduler. The
    term can also refer to a single aggregate view of multiple
    distributed resource managers, but that is not the topic of this
    document.} for cycling environmental forecast suites containing many
    interdependent scientific models and associated data processing
    tasks.\footnote{A {\em task} is any group of processes treated as a
    single entity for scheduling purposes.} Cylc is internally
    self-organising:\footnote{Prior to version 3.0 cylc's suite design
    interface reflected this too - a suite was defined by a set
    of distinct ``task definition files'' that specified inputs and
    outputs (and a few other parameters) for each task; now one specifies
    the suite dependency graph up front.} its novel scheduling algorithm
    allows an evolving pool of
    task proxy objects to resolve dependencies amongst themselves so
    that correct scheduling emerges naturally at run time.  Cylc does
    not group tasks artificially by forecast cycle,\footnote{For our
    purposes a {\em forecast cycle} comprises all tasks with a common
    {\em cycle time}, i.e.\ the analysis time or nominal start time of a
    forecast model, or that of the forecast model(s) associated with the
    other tasks.} and its task proxies are individually self-spawning
    (i.e.\ there is no ``global suite forecast cycle'') so that tasks
    from multiple forecast cycles can run at once to the full extent
    allowed by the real intercycle dependencies. This matters in
    particular whenever the external driving data\footnote{Forecast
    systems are typically driven by observational data and/or timely
    model fields from an external forecasting system.} for upcoming
    cycles are available in advance: cylc suites can catch up from
    delays very quickly, parallel test suites can be started or
    restarted behind the main operation to catch up quickly, and one can
    likewise achieve greater throughput in historical case studies. The
    usual sequence of distinct forecast cycles emerges naturally as a
    suite catches up to real time operation.  Cylc is easily interfaced
    to existing tasks and is extremely flexible.  Suites can be stopped
    and restarted in any state of operation, and they dynamically adapt
    to insertion and removal of tasks, and to delays or failures in
    particular tasks or in the external environment (tasks not directly
    affected will carry on cycling as normal while the problem is
    addressed, after which time the affected tasks will catch up as
    quickly as possible). Cylc's handling of forecast model restart
    dependencies, and ability to recursively remove entire sub-trees of
    tasks, allows continued operation, with very little operator
    intervention, over major failures that require skipping cycles in
    the driving models. Cylc can control tasks distributed across a
    heterogenous network. Cylc has comprehensive graphical and command
    line interfaces, and many advanced userspace features: job
    databases, simulation mode, suite validation, dependency graph
    plotting, centralised alert hooks for failures and timeouts,
    cryptographically secure suites, \ldots

    %\footnote{Cylc also
    %enables new modes of real time operation, for example a catchment
    %river model that runs hourly assimilating real time stream flow
    %observations and using the {\em most recent} 6-hourly precipitation
    %forecast - see EcoConnect, below).} 
\end{abstract}


