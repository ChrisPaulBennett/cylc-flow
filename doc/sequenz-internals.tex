%\documentclass[11pt,a4paper]{report}
\documentclass[11pt,a4paper]{article}

\usepackage{listings}
\usepackage{amsmath}

\lstset{language=Python}

\title{An Optimal Dynamic Metascheduler for Continuously Cycling
Multi-Model Forecasting Systems}

\author{Hilary Oliver, NIWA}

\begin{document}

\maketitle
\tableofcontents

\begin{abstract}

Metascheduling is the process of determining when each task in a system
of tasks with dependencies is ready to be submitted to the batch queue
scheduler; i.e.\ when tasks could execute given available computing
resource because their prerequisites have been satisfied by other tasks
in the system*. This paper describes an extremely flexible
metascheduling algorithm that ensures correct order of execution in a
complex continuously cycling multi-model forecasting system without any
need to dictate execution order or to explicitly define the dependencies
between tasks. Each task simply registers its own prerequisites, without
reference to other tasks, and task proxy objects then negotiate
dependencies dynamically, as outputs are reported, so that correct
execution order emerges naturally at run time. The method is optimal in
that tasks can be submitted the instant their prerequisites are
satisfied regardless of their associated ``forecast cycle'' or any other
consideration. The entire system can therefore run over historical case
study periods, or catch up to real time after a delay, with maximum
throughput: tasks from many different forecast cycles can run
simultaneously where dependencies allow. The absence of explicit task
sequencing logic or prescribed dependencies makes the system very
flexible. New tasks can be added and existing ones deleted or switched
on and off without any change to the algorithm, and partial parallel
test systems that feed off the main operation can be created very
easily. Our metascheduling controller, {\em Sequenz}, can be interfaced
to existing tasks (models etc.) with minimal overhead. It is written in
object oriented Python and uses the Python Remote Object protocol ({\em
Pyro}) to control tasks on multiple platforms at once. 

(*) The term ``metascheduler'' can also be used to refer to ``a single
aggregated view of multiple Distributed Resource Managers''.

\end{abstract}

\section{Introduction}

Environmental forecasting systems typically consist of one or more
forecast models and any number of pre- and post-processing tasks. Each
task has an associated "forecast reference time"; this is the nominal
analysis time or start time for the forecast models themselves, and for
pre- and postprocessing tasks it is the reference time of the associated
forecast tasks. Each task has a regular series of reference times at
which it executes (e.g. 00, 06, 12, and 18 hours every day), but not
necessarily the series for every task in the system (a global
atmospheric model that runs only twice daily, for instance, may drive
regional models that run four times daily). For a set of tasks with the
same reference time, intertask dependencies dictate a pattern of
execution that forms a Directed Acyclic Graph (each task may take input
from zero or more upstream tasks and may provide input to zero or more
downstream tasks). However, forecasting systems are continuously cycling
and there are additional dependencies between tasks with different
reference times.  Principally, forecast models ususally depend on their
own previous instance because each forecast generates a ``background
model state'' for the next. Post processing tasks, on the other hand,
don't depend on their own previous instances, but they necessarily
depend on cotemporal forecast tasks that do. In addition, there may be
some tasks that depend on non-cotemporal tasks of another class (for
instance, we have an hourly river catchment model that takes input from
the {\em most recent} 6-hourly regional weather model, which it is
allowed to run ahead of to a limited extent (while it assimilates
incoming real time streamflow observations). Finally, there will be some
pre-processing tasks that have no upstream dependencies but are
constrained to wait on some external event (such as incoming real time
observations), and there may be some that have no prerequisites and no
constraints (e.g. ocean tide forecasts, for which all reference times
could ``go off at once'' if not artificially constrained in some way).  

There is at least one generic metascheduler that has long been
available, Condor's DAGMan (ref:). However, it appears to be for "one
off" dependent job sets and requires dependencies to be explicitly
specified.  To the author's knowledge there has been nothing published
on metascheduling controllers for continuously cycling forecast systems,
probably because the level of complexity has led most forecasting
centres to develop in-house solutions that are highly specific to their
particular task set and therefore not widely applicable. In addition,
these in-house solutions invariably simplify by insisting that only a
single ``forecast cycle'' can run at a time (i.e. all tasks with one
reference time must complete before any from the next can start). This
can arguably be justified on the grounds that it is usually adequate for
real time operation for which there is a gap between cycles anyway (but
see **) as the system waits on arrival of real time observational data,
or possibly on forecast fields from an external system that is so
constrained. However, this is a significant impediment when catching up
to real time operation after a delay, or when running the entire system
over historical case study periods; in these situations external input
for one or more cycles is immediately available and consequently a lot
of parallelism is possible: tasks from multiple reference times can run
at once where dependencies allow. The biggest gain here is that
successive instances of the driving (upstream) forecast models can, in
principle, run without delay, regardless of ongoing downstream
processing (dependent models and post processing) associated with
previous forecasts. Similarly, but constrained but cotemporal upstream
dependencies, the downstream forecast models need not wait for an entire
cycle to finish either, and in, and in some cases successive instances
of the same post-processing task may be able to overlap (these have no
previous-instance dependence; overlap can occur when a post processing
task takes longer to run than the delay between its upstream model
starting and it starting).

(** except: topnet)

Metascheduling is the processing of determining, for a set of tasks that
depend on one another, when each task is ready to be sent to the batch
queue scheduler; i.e. when it is ready to execute given available
hardware resource. Some modern batch queue schedulers now have a
rudimentary ability to handle depedencies but they are not nearly
sophisticated enough to provide optimal control of a complex
continuously cycling multi-model forecasting system, and typical
in-house solutions are highly system-specific and inflexible. 


Metascheduling is the process of determining when a task\footnote{A {\em
task} is any group of processes to be treated as a single entity for
scheduling purposes (and which would generally be controlled by a single
top level script or program).} is {\em ready to execute} in the sense
that all of its prerequisites have been satisfied by other tasks in the
system, at which point it may be submitted to a normal batch queue
scheduler for allocation to available computing resource. Each task in
such a system may depend on one or more ``upstream'' tasks for input,
and may generate output for use by one or more ``downstream'' tasks.

The National Institute of Water and Atmosperhic Research (NIWA), New
Zealand, has developed an operational environmental forecasting system,
EcoConnect, which consists of chained global sea state, regional
atmospheric (data assimilating), sea state, storm surge, river flow, and
tide models, driven by the UK Met Office operational global atmospheric
model, and many associated pre- and post-processing, product generation,
and archiving tasks, etc. The system runs on heterogenous distributed
hardware, and each task runs between 1 and 24 times per day, depending
on the task and the model(s) it is associated with. Most tasks depend
primarily on one or more of their cotemporal peers and, in the case of
the scientific models, their immediate predecessor of the same task type
(because forecast models generally start from a ``background state''
generated by the previous forecast) but some have more complex
dependencies on previous forecast cycles\footnote{For example, the
catchment model, TopNet, runs hourly, driven by precipitation forecasts
from the {\em most recent} 12-hourly regional atmospheric run, {\em and}
assimilates stream flow observations that come in hourly (real time).
However, precipitation forecasts are delayed by several hours with
respect to real time, and possibly longer if system delays occur, so we
allow TopNet to run ahead of the atmospheric model by more than the
minimum 0-12 hours in order to make the best use of new streamflow
observations.}.  Finally, because of limited computational resource we
have to be able to run tasks from multiple forecast cycles at once,
where dependencies allow, for maximum system throughput after downtime,
or for historical case study runs for which all external input data is
already available.

\subsection{Traditional Solutions}

Controlling such a system is largely a matter of determining when each
task's inputs, generated by other tasks in the system, are ready (when
they are, allocation to available computational resource can be handled
by a normal batch queue scheduler).  This could be done by checking for
the existence of required inputs directly, or by monitoring the state of
the other tasks that are known to provide the inputs in each case (are
they finished yet?). It is not easy to do this in a flexible and
extensible way, however, because the control diagram will, in general,
take the form of a \textbf{Directed Acyclic Graph}: multiple parallel
streams of execution that branch (when one task generates output for
several downstream task) and merge (when one task takes input from
several upstream task), and the detailed properties of this network
depend entirely on the particular inputs and outputs of the current
configured task set.  A typical in-house solution is a simple control
program that executes a user-specified task list (assumed to be
correctly ordered already) in a linear sequence, or some form of highly
specific Finite State Machine that enforces a predetermined (but
possibly non-linear) order of operations using hard coded sequencing
logic: {\em if A has generated file X, and B has finished, then start
executing C}, etc. Unfortunately, the former method is highly
inefficient because it is likely that many tasks could execute in
parallel, and the latter quickly becomes very convoluted and inflexible
as system complexity increases. In addition, forecasting control systems
typically demand that all tasks from one cycle complete before starting
the next cycle, even though any task can run, in principle, as soon as
its own prerequisites are satisfied, regardless of whether any tasks
from previous cycles are still running or yet to start.  While this
simplification is reasonable in normal real time operation where time is
spent between cycles waiting on external input (e.g. new observational
data), it is a significant impediment when external input for multiple
cycles is already available, such as when catching up after system
downtime in a resource constrained computational environment, or when
running the entire system over historical case study periods.
Attempting to relax this constraint to achieve ``asynchronous''
behaviour in a hard coded finite state machine, however, dramatically
exacerbates the complexity of 


\subsection{Dynamic Metascheduling}

Dynamic Metascheduling is an object oriented approach to the problem.
Within the control program each task is represented by a self-contained
{\em task object} that knows nothing about other tasks in the system but
does know its own prerequisites (inputs) and outputs, and can interact
with other tasks to negotiate dependencies so that correct sequencing
emerges naturally at run time.  The entire sequencing problem is now
contained within the task interaction step, which is almost trivial
because it is completely indiscriminate: whenever any task changes
state, {\em each task} asks {\em all the others} if they can satisfy its
prerequisites with their completed outputs (it does not matter that most
of these interactions are destined to be fruitless). The control program
thus remains simple and generic, regardless of the number of tasks or
the complexity of their interdependencies; it simply manages a set of
tasks that are all individually configured as if they were to run in
isolation.\footnote{The system manager does of course have to ensure
that the configured task pool is self consistent, i.e. that each task's
prerequisites will be satisfied by some other task(s) in the system.}
The total absence of explicit sequencing logic makes this method
extremely flexible and extensible.\footnote{To extend the system, one
simply derives a new class definition that lists the new task's
prerequisites and outputs. The new task will automatically run at the
right time, i.e. when its prerequisites have been satisfied by some
other task(s) in the system.}

Dynamic sequencing can be viewed as a simulation of an interacting task
set in which the state of each {\em simulee} is tied to, and can
influence, that of the real task it represents.  This suggests a
powerful dummy mode of operation in which the simulation is divorced
from reality by replacing external tasks with an external dummy program
that masquerades as a real task by ``completing'' each of its outputs in
turn (i.e. it reports they are complete because, as far as the
controller is concerned, task outputs are just {\em messages}; more on
this below).  The control program cannot distinguish this from real
operations, so the dummy mode allows complete testing of the control
system for a given configured task set, without running any real tasks.


\subsection{{\em Sequenz}}

A dynamic sequencing controller would clearly be most naturally
implemented in an object oriented programming language, which provides
the tools for object construction and interaction.  We must also provide
the means for task objects to launch their external tasks when their
prerequisites are satisfied (e.g.  by submission to a batch queue
scheduler on a task-dependent target machine), and for synchronization
of the internal states of task objects and the external tasks they
represent, via remote method calls or something similar.  Finally, the
generic description above assumes that task objects exist whenever they
are needed (for external task control and for interacting with other
dependent tasks), so we must devise a task management scheme governing
the creation and destruction of task objects, to ensure that this is the
case.

NIWA's EcoConnect Forecasting System, described above, is controlled by
an asynchronous dynamic sequencing implementation called {\em Sequenz},
implemented in object oriented Python and using the Python Remote Object
protocol ({\em Pyro}) for direct network communication between task
objects and the tasks they represent. It allows tasks from many
different forecast cycles to run at once, where dependencies allow,
transitions seamlessly from ``catchup'' or case study mode (where all
external input is available from the outset and much asynchronous
activity is therefore possible, e.g. after system downtime; in this
case) to normal real time operation, tasks can be switched on and off
while the controller is running, new tasks added without any change to
the main program, and the system can be restarted in arbitrarily complex
states after downtime.  It also features the powerful dummy mode
described above, for testing, and allows easy construction of ad hoc
parallel test systems feeding off the main operational system. Pyro also
enables efficient RPC-based remote control and system monitoring by
external programs.  Sequenz can be interfaced to existing models and
tasks with minimal overhead. 


\subsubsection{Applicability}

The dynamic sequencing concept is very general and could in principle be
implemented for any set of interdependent tasks. Sequenz, however, is
somewhat specialized toward cycling forecast systems, such as
EcoConnect, in which each task has an associated {\em reference time}
(generally the nominal forecast start time, or {\em analysis} time, of
the associated forecast model) that all task prerequisites and outputs
depend on in some way. and a set of valid times for each task type (e.g.
the atmospheric model might run at 00, 06, 12, and 18 UTC each day,
while the river model runs hourly).  One-off task pools with no
particular time dependence, however, are much simpler than this, and it
would not be difficult to strip all reference time handling from the
program.

In addition, EcoConnect operates in a well defined environment so that,
for example, each task knows exactly what its input files look like
(filenaming conventions) and where to get them from (e.g. {\em task X}'s
output directory). Consequently, for file-based prerequisites, the
controller does not need to know the file's actual location or check for
its existence, because the external tasks already do that. It could,
however, easily be made to pass file locations between tasks so that all
input/output filenames and paths could be defined centrally in the
control program itself.  


\subsection{Alternatives}

As far as the author is aware, there are currently no other systems
available that can do general {\em asynchronous} sequencing, as
described in this paper.

In the planning stage, in addition to considering a Finite State Machine
design, as discussed above, we also considered new generation batch
queue schedulers (which now allow simple job dependencies), and use of
some kind of generic control framework.  Schedulers are still too
rudimentary for our purposes, however, and no generic control framework
that was obviously well suited came to light. In any case, the dynamic
sequencing design described in this paper simplifies the problem to such
an extent that it is unlikely any generic framework could compete.  


\section{{\em Sequenz} Implementation}

To implement dynamic sequencing we need a specific {\em task management}
scheme that controls when tasks will be created and destroyed, and so
on. There are many options here; for instance: should tasks be created
all at once, in cotemporal batches, or one at a time as their
predecessors finish? The consequences of these choices must be evaluated
carefully, which isn't easy in light of the inherent complexity of
asynchronous operation. At the simpler end of the task management
spectrum we could, for example, create a whole month's worth of tasks at
once and let them interact until everything runs to completion.  Nothing
would run out of sequence, {\em but} system monitoring would be
difficult because of the sheer number of tasks involved, the end of the
month would present an artificial barrier to operations, tasks that lack
prerequisites would all want to run at once, and system restarts would
be overly complicated. 

\subsection{Main Algorithm}

The algorithm below operates on a single pool of interacting task
objects, has simple start up and continuous operation, and is relatively
easy to monitor (task objects exist by the time they are needed but not
for too long before that, and not for too long after they are finished). 

The simplicity of the dynamic sequencing algorithm is clear from the
following code listing, taken directly from the main program:

{\small
\noindent
\rule{5cm}{.2mm}
\begin{lstlisting}
# (startup initialization code omitted)

while True: # MAIN LOOP

   if task_base.state_changed:
       # PROCESS ALL TASKS whenever one has changed state
       # as a result of a remote task message coming in: 
       # interact, run, create new, and kill spent tasks
       #---
       task_pool.process_tasks()
       task_pool.dump_state()
       if task_pool.all_finished():
           clean_shutdown( "ALL TASKS FINISHED" )

    # REMOTE METHOD HANDLING; handleRequests() returns 
    # after one or more remote method invocations are 
    # processed (these are not just task messages, hence 
    # the use of task_base.state_changed above).
    #---
    task_base.state_changed = False
    pyro_daemon.handleRequests( timeout = None )

# END MAIN LOOP
\end{lstlisting}
}

\subsection{Details}

\subsubsection{Startup and Initialization}

An initial reference time and list of task object names are read in from
the config file, then each task object is created at the initial
reference time {\em or} at the first subsequent reference time that is
valid for the task type. Optionally, we can tell the controller to
reload the current state dump file (which may have been edited); this
will override the configured start time and task list. After startup,
new tasks are created only by {\em abdication} (below).

An initial run through the {\em task processing} code, by virtue of the
fact that the main loop starts with task processing, causes tasks with
no prerequisites (e.g. {\em downloader}) to enter the {\em running}
state and launch their external tasks immediately. Otherwise ({\em or}
if there are no tasks that lack prerequisites) nothing will happen.



\subsubsection{Creation of New Tasks}

New tasks are created by abdication, i.e. create $foo(T\negmedspace
+\negmedspace 1)$ if $foo(T)$ is {\em finished}.  Task abdication
ensures that $foo(T\negmedspace +\negmedspace 1)$ won't run before
$foo(T)$ finishes, without imposing explicit intercycle prerequisites
that would require special treatment at startup (when there is no
previous cycle).  It also ensures that tasks with no prerequisites, e.g.
{\em downloader} and {\em nztide}, won't all try to run at once.
Tasks are not deleted immediately on abdication (see below).


\subsubsection{Task Interaction} 

Each task keeps track of which of its postrequisites are completed, and
asks the other tasks if they can satisfy any of its prerequisites.  The
fact that task objects do not need to know who is supposed to satisfy
their prerequisites (because they can ask, indiscriminately, every other
task in the system) makes the task interaction (sequencing!) algorithm
almost trivial. The fact that most of these interactions are fruitless
is of no consequence. 

{\small
\noindent
\rule{5cm}{.2mm}
\begin{lstlisting}
class task_pool( Pyro.core.ObjBase ):
    # ...
    def interact( self ):
        # get each task to ask all the others if 
        # they can satisfy its prerequisites
        #--
        for task in self.tasks:
            task.get_satisfaction( self.tasks )
    # ...
\end{lstlisting}
}

\subsubsection{Running Tasks}

Each task object can launch its associated external task, and enter the
{\em running} state if its prerequisites are all satisfied, any existing
older tasks of the same type are already {\em finished}, and fewer than
{\em MAX\_ RUNAHEAD} finished tasks of the same type still exist (this
stops tasks with no prerequisites from running ahead indefinitely).

\subsubsection{Pyro Remote Method Calls}

The Pyro request handling loop executes remote method calls coming in
from external tasks, and returns after at least one call was handled.
Pyro must be run in non-default single-threaded mode (see Appendix
\ref{pyro-appendix}).

\subsubsection{Dumping State} 

The current state (waiting, running, or finished) of each task is
written out to the {\em state dump file}.  This provides a running
snapshot of the system as it runs, and just prior to shutdown or
failure. The controller can optionally start up by loading the state
dump (which can be edited first). Any 'running' tasks are reloaded in
the 'waiting' state.

\subsubsection{Removing Spent Tasks} 

A task is spent if it finished {\em and} no longer needed to satisfy the
prequisites of any other task. Most tasks are only needed by other
cotemporal downstream tasks; these can be removed when they are finished
{\em and} older than the oldest non-finished task. For rare cases that
are needed by tasks in later reference times (e.g. nzlam post
processing: multiple hourly topnet tasks need the same most recent
previously finished 06 or 18Z nzlam post processing task), each
non-finished task reports its {\em cutoff reference time} which is the
oldest reference time that may contain tasks still needed to satisfy its
own prerequisites (if it is waiting) or those of its immediate
post-abdication successor (if it is running already), then the task
manager can then kill any finished tasks that are also older than the
oldest task cutoff time.

\subsubsection{Removing Lame Tasks} 

Tasks that will never run (because their prerequisites cannot be
satisfied by any other task in the system) are removed from the {\em
oldest batch} of tasks.  If not removed they would prevent the spent
task deletion algorithm from working properly. Lame tasks can only be
detected in the oldest task batch; in younger batches some tasks may yet
appear as their predecessors abdicate.

Lame tasks are abdicated rather than just deleted, because their
descendents will not necessarily be lame: e.g. if the system is started
at 12Z with topnet turned on, all topnet tasks from 12Z through 17Z will
be valid but lame, because they will want to take input from a
non-existent nzlam\_post from 06Z prior to startup. However, the
presence of lame tasks may indicate user error: e.g. if you forget
to turn on task type $foo$ that supplies input to task type $bar$,
any instance of $bar$ will be lame.


\subsection{Dummy Mode}

Dummy mode allows complete testing of the control system without running
any real external tasks\footnote{The only difference between dummy mode
and real operation, as far as the controller is concerned, is that
external dummy tasks are not delayed by resource contention.}. When it
is ready to run, a task object will launch an external dummy program
that (i) gets a list of postrequisites from the parent task object and
then (ii) reports back that each one is satisfied at the (estimated)
right time relative to an accelerated dummy clock. Dummy tasks therefore
complete in approximately the same dummy clock time as the real tasks do
in real time. An initial dummy clock offset relative to the initial
reference time can also be specified, which allows simulation of the
transition between catchup and real time operation. Log messages are
stamped with dummy clock time instead of real time.

The same script is used for all external dummy tasks but it has special
behaviour in certain cases: the dummy downloader ``waits for incoming
files'' until 3:15 past its reference time, and the dummy topnet ``waits
for stream flow data'' until 0:15 past its reference time.

The dummy clock can be bumped forward a number of hours by remote
control, while the system is running. This affects the postrequisite
timing of running tasks correctly, but if it causes a running task to
finish immediately the next task in line will still start from the
beginning no matter how big the bump.


\appendix

\section{Essential OOP Concepts}

The simplicity of the dynamic sequencing implementation in {\em sequenz}
is critically dependent on the {\em polymorphic} nature of the {\em task
objects} in the program.  This section contains a minimal introduction
to these Object Oriented Programming concepts.  Refer to any OOP
reference for more detail.

\subsection{Classes and Objects}

A {\em class} is essentially a generalisation of {\em data type} to
include {\em behaviour} (i.e. functions or {\em methods}) as well as
state.  {\em Objects} are more or less self contained {\em instances} of
a class. For example, a $shape$ class could define a $position$ data
member that describes the location of each shape object, a $move()$
method that causes a shape object to alter its position, and a $draw()$
method that causes it to display itself in the right place on screen.

\subsection{Inheritance}

A {\em derived class} or {\em subclass} inherits the properties (methods
and data members) of a {\em base class}. It can also {\em override}
specific base class properties, or add new properties that aren't
present in the base class. Calling a particular method on an object
invokes the object's own class method if one is defined, otherwise the
immediate base class is searched, and so on down to the root of the
inheritance graph. 

For example, we could derive a $circle$ class from $shape$, adding a
`radius' data member and overriding the $draw()$ to get circle objects
to display themselves as actual circles.  Because we didn't override the
$move()$ method, calling $circle.move()$ would invoke the base class
method, $shape.move()$. 


\subsection{Polymorphism}

Polymorphism is the ability of one type to appear as and be used like
another type.  In OOP languages with inheritance, this usually refers to
the ability to treat derived/sub-class objects as if they were members
of a base class.  In particular, a group of mixed-type objects can all
be treated as members of a common base class. For example, we could
construct a list of $shape$ objects from $circles$, $triangles$, and
$squares$; calling $[list member].draw()$ will invoke the right derived
class $draw()$. This is a very powerful mechanism because {\em it allows
unmodified old code to call new code}: if we later derive an entirely
new kind of shape ($hexagon$, say) with it's own unique behaviour, the
existing program, without modification, will process the new objects in
the proper hexagon-specific way.


\section{Threading in Pyro} \label{pyro-appendix}

With Pyro in {\em single threaded mode}, \verb#handleRequests()# returns
after {\em either} a timeout has occurred {\em or} at least one request
(i.e.  remote method call) was handled. With \verb#timeout = None# this
allows us to process tasks {\em only} after remote method invocations
come in.  Further, we can detect the remote calls that actually change
task states, and thereby drop into the task processing code only when
necessary, which minimizes non-useful output from the task processing
loop (e.g. in dummy mode there are a lot of remote calls on the dummy
clock object, which does not alter tasks at all). 

In {\em multithreaded mode}, \verb#handleRequests()# returns immediately
after creating a new request handling thread for a single remote object
and thereafter remote method calls on that object come in asynchronously
in the dedicated thread. This is not good for the dynamic sequencing
algorithm because tasks are only set running in the task processing
block which can be delayed while \verb#handleRequests()# blocks waiting
for a new connection to be established even as messages that warrent
task processing are coming in on existing connections. The only way
around this is to do task processing on \verb#handleRequests()# timeouts
which results in a lot of unnecessary task processing when nothing
important is happening.


\section{YET TO BE DOCUMENTED}

\begin{itemize}
 \item usage
 \item logging
 \item debugging
 \item adding new task definitions
 \item system monitors
 \item remote control: 
    \begin{itemize}
    \item clean shutdown of pyro
    \item bumping the dummy clock forward
    \end{itemize}
 \item external task messaging interface script
 \item dummying out tasks in real mode
 \item how to set up partial parallel test systems
\end{itemize}

\section{Miscellaneous Notes}

(To be incorporated into the main documentation, or deleted).

\subsection{catchup mode}

Note that ``correct model sequencing'' is not equivalent to ``orderly
generation of products by reference time''.  Nzlam can run continuously,
regardless of the downstream processing that depends on it.

Catchup mode is now task-dependent, rather than a property of the whole
system (as it is if only tasks from the same reference time can run at
once).  Where this matters, it can be detected by the relevant external
task. E.g. if the external topnet(T) task starts up at real time t
greater than the stream flow data time for T (i.e. $T+15$ min), i.e. the
required stream flow data is already available, then we're still in
catchup. If, on the other hand, the topnet task finds that it has to
wait for its stream flow data time to arrive, then we're caught up to
real time.  This matters because topnet is allowed to run ahead by a
different amount of time depending on whether we're in catchup mode or
not.


\subsection{controlling when a task executes}

\begin{itemize}
 \item  prerequisites
 \item artificial prerequisites (e.g. make nztide depend on nzlam)
 \item delayed instantiation (a task can't run if it doesn't exist yet).
 \item other contraints based on, for example, the number of previous
 instances that still exist in the system.
\end{itemize}


\subsection{fuzzy prequisites}

{\em Exact prerequisites} (most tasks): times are specified exactly,
relative to the task's own reference time.  E.g. {\em file foo\_{T}.nc
ready} where T is the task's reference time.

{\em Fuzzy prerequisites} (topnet): a time boundary is specified
relative to the task's own reference time; any task with a reference
time greater than or equal to the boundary time can satisify the
prerequisite.

\subsection{Task Messaging}

To the task objects, outputs are just {\em messages} indicating that the
external task has reached a significant waypoint, such as generation of
a particular output file.

\section{To Do}

\begin{itemize}

\item
 retrofit proper exception handling throughout (currently used sparsely and 
  inconsistently)

\item
 potential bug in the free task class: will never run if too many
 previous instances exist that are newer than the task deletion cutoff

\item
 remote logging from messaging and dummy task scripts, in case the
 controller itself is dead.

\end{itemize}

\end{document}
