\section{Roadmap}

Several planned future developments in Rose and Cylc may have an impact on
suite design.

\subsection{List Item Override In Site Include-Files}
\label{List Item Override In Site Include-Files}

A few Cylc config items hold lists of task (or family) names, e.g.:

\lstset{language=suiterc}
\begin{lstlisting}
[scheduling]
    [[special tasks]]
        clock-trigger = get-data-a, get-data-b
    #...
#...
\end{lstlisting}

Currently a repeated config item completely overrides a previously set value
(apart from graph strings which are always additive). This means a site
include-file (for example) can't add a new site-specific clock-triggered task
without writing out the complete list of all clock-triggered tasks in the
suite, which breaks the otherwise clean separation into core and site files.

{\em In the future we plan to support add, subtract, unset, and override
semantics for all items - see \url{https://github.com/cylc/cylc/issues/1363}}.

\subsection{UM STASH in Optional App Configs}
\label{UM STASH in Optional App Configs}

A caveat to the advice on use of option app configs in Section~\ref{Optional
App Config Files}: in general you might need the ability to turn off or modify
some STASH requests in the main app, not just add additional site-specific
STASH. But overriding STASH in optional configs is fragile because STASH
namelists names are automatically generated from a {\em hash} of the precise
content of the namelist. This makes it possible to uniquely identify the same
STASH requests in different apps, but if any detail of a STASH request changes
in a main app its namelist name will change and any optional configs that refer
to it will become divorced from their intended target.

Until this problem is solved we recommend that:

\begin{itemize}
  \item All STASH in main UM apps should be grouped into sensible {\em
    packages} that can be turned on and off in optional configs without
    referencing the individual STASH request namelists.
  \item Or all STASH should be held in optional site configs and none in the
    main app. Note however that STASH is difficult to configure outside of
    \lstinline=rose edit=, and the editor does not yet allow you to edit
    optional configs - see \url{https://github.com/metomi/rose/issues/1685}.
\end{itemize}

\subsection{Modular Suite Design}

The modular suite design concept is that we should be able to import common
workflow segments at install time rather than duplicating them in each suite:
\url{https://github.com/cylc/cylc/issues/1829}. The content of a suite module
will be encapsulated in a protected namespace to avoid clashing with the
importing suite, and selected inputs and outputs exposed via a proper
interface.

This should aid portable suite design too by enabling site-specific parts of a
workflow (local product generation for example) to be stored and imported
on-site rather than polluting the source and revision control record of
the core suite that everyone sees.

We note that this can already be done to a limited extent by using 
\lstinline=rose suite-run= to install suite.rc fragments from an external
location. However, as a literal inlining mechanism with no encapsulation or 
interface, the internals of the ``imported'' fragments would have to be
compatible with the suite definition in every respect.

See also~\ref{Monolithic Or Interdependent Suites} on modular {\em systems of
suites} connected by inter-suite triggering.
