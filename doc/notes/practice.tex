
A running scheduler uses the state dump and logging directories that
depend on the registered name under which the system is running.
After a restart, the same state dump and logging directories will
continue to be used, {\em unless} you do a practice mode restart. 

On a restart, except in practice mode, cylc assumes that any tasks
recorded as unfinished (i.e. submitted, running, or failed) need to be
re-run, so it resets them to the waiting state unless you specify the
'--no-reset' option. If you do specify no-reset, be aware that any task
started in the 'running' state will need to be manually reset to either
waiting (for a re-run) or finished, if you know that the real external
task has completed successfully.

In practice mode, the registered system's state dump is used to
initialise the new practice system, but the new system uses its own
state dump and logging directories - i.e. it has no effect on the
original system. You can use a practice mode restart to check that
intended run-time changes involving task deletion or insertion will have
the desired effect, before inflicting them on the real system.
Unfinished tasks are not reset to waiting at startup in practice mode,
because it is assumed that you are using practice mode to work out how
to recover from a failure. 
