\section{Object Oriented Programming}
\label{ObjectOrientedProgramming}

Cylc relies heavily on Object Oriented Programming (OOP) concepts,
particularly the {\em polymorphic} nature of the task proxy objects.
An absolutely minimal explanation of this follows; 
please refer to an OOP reference for more detail.

A {\bf class} is a generalisation of data type to include behaviour
(i.e.\ functions or methods) as well as state. 

%For example, a $shape$ class could define a $position$ data member to
%hold the location of a shape object, a $move()$ method that by which
%a shape object can alter its position, and a $draw()$ method that
%causes it to display itself on screen.

An {\bf object} is a more or less self contained specific instance
of a class. This is analagous to specific integer variables being 
instances of the integer data type.

A {\bf derived class} or {\bf subclass} {\em inherits} the properties
(methods and data members) of its parent class. It can also override
specific properties, or add new properties that aren't present in the
parent. Calling a particular method on an object invokes the object's
own method if one is defined, otherwise the parent class is searched,
and so on down to the root of the inheritance graph. 

%For example, we could derive a $circle$ class from $shape$, adding a
%`radius' data member and overriding the $draw()$ to get circle objects
%to display themselves as actual circles.  Because we didn't override the
%$move()$ method, calling $circle.move()$ would invoke the base class
%method, $shape.move()$. 


{\bf Polymorphism} is the ability of one type to appear as and be used
like another type.  In OOP languages with inheritance, this usually
refers to the ability to treat derived/sub-class objects as if they were
members of a common base class. In particular, a group of mixed-type
objects can all be treated as members of a common base class. 
%For example, a group of %$circles$, $triangles$, and $squares$ could 
%be manipulated by code designed entirely to handel $shapes$; calling
%$[shape].draw()$ will invoke the right derived class $draw()$ method. 
This is a powerful mechanism because it allows existing old code,
without modification, to manipulate new objects so long as they 
derive from the original base class.
%If we later derive an entirely new kind of shape ($hexagon$, say) with
%it's own unique behaviour, the existing program, without modification,
%will process the new objects in the proper hexagon-specific way.  

In cylc, all task proxy objects are derived from a base class that 
embodies the properties and behaviour common to all task proxies. 
The scheduling algorithm works with instances of the base class so that
any current or future derived task object can be handled by the program
without modification (other than deriving the new subclass itself).

\subsection{Single- or Multi-Threaded Pyro?}
\label{Single-orMulti-ThreadedPyro?}

In single threaded mode Pyro's \lstinline=handleRequests()= returns
after at least one request (i.e.\ remote method call) was
handled, or after a timeout. Using \lstinline|timeout = None| 
allows us to process tasks only when remote method invocations
come in.  Further, we can detect the remote calls that actually change
task states, and thereby drop into the task processing code only when
necessary, which eliminates a lot of extraneous output when debugging
the task processing loop (e.g.\ in dummy mode there are a lot of remote
calls on the dummy clock object, which does not alter tasks at all). 

In multithreaded mode, \lstinline=handleRequests()= returns immediately
after creating a new request handling thread for a single remote object,
and thereafter remote method calls on that object come in asynchronously
in the dedicated thread. This is not good for cylc's scheduling
algorithm because tasks are only set running in the task processing
block which can be delayed while \lstinline=handleRequests()= blocks waiting
for a new connection to be established, even as messages that warrant
task processing are coming in on existing connections. The only way
around this seems to be to do task processing on \lstinline=handleRequests()=
timeouts which results in a lot of unnecessary processing when nothing
important is happening.

Addendum, we now use a timeout on \lstinline=handleRequests()= because
contact tasks can trigger purely on the wall clock, so we delaying task
processing when no messages are coming in may prevent these contact
tasks from triggering.   So\dots we may want to revist Multithreading\dots

%\subsubsection{Handling File Dependencies: Possible Alternative Method}
%
%In principle extra information could be attached to cylc output
%messages so that actual file locations could be passed dynamically from
%to whichever tasks use the output. Cylc currently cannot do this (you
%can put actual file locations in the messages, but the receiver has to
%have the exact matching message and therefore would have to know the
%location in advance). This is a possible future development, but is 
%probably not worth the effort because configuring the external tasks 
%to report this information takes more effort than putting the same
%information into the cylc task definition files. The cylc setup
%would remain entirely context-independent, which is nice, and would
%automatically pass on changes to the external input / output config of
%the system.

%\subsection{Unusual Task Behaviour}
%\label{UnusualTaskBehaviour}
%
%If you require task behaviour that cannot be represented in the current 
%task definition files you will need to derive a new task class manually.
%Use the auto-generated task classes as a starting point. Raw Python 
%task class definitions can be kept in the suite taskdef sub-directory
%alongside the taskdef files; they will be copied verbatim to the
%\lstinline=configured= sub-directory when the suite is configured.
%
%Out of the entire EcoConnect operation, only the highly unusual
%scheduling behaviour of the TopNet river model requires a custom task
%class (it keeps up with real time streamflow observations and uses
%the {\em most recent} regional weather forecast output). 

%\subsubsection{Fuzzy Prerequisites}
%
%EcoConnect's Topnet model (mentioned just above) runs hourly and
%triggers off the most recent regional weather forecast available.
%The cycle point interval between the two tasks can vary. This makes 
%use of cylc's {\em fuzzy prerequisites}, which the task definition
%parser is not currently aware of (hence the custom Python taskdef).

%\subsection{Task Prerequisites And Outputs}
%\label{TaskPrerequisitesAndOutputs}
%
%Cylc's scheduling algorithm matches one task's completed outputs with
%another's unsatisfied prerequisites
%(Section~\ref{TheCylcSchedulingAlgorithm}).  
%
%Internally, these prerequisites (which must be satisfied before the task
%can run) and outputs (that must be be completed as the task runs) take
%the form of {\em literal text strings - messages that running tasks 
%send to their proxy objects inside the scheduler}.
%
%\begin{myitemize}
%    \item A task proxy considers a registered output ``completed''
%        if it has received a matching message from its external task.
%
%    \item A task proxy considers a registered prerequisite ``satisfied''
%        if another task proxy reports that it has a matching completed
%        output.
%
%\end{myitemize}
%
%\subsubsection{Cycle Point}
%
%{\em Prerequisites and outputs should always contain a cycle point} to
%distinguish between different instances of a task (at different 
%forecast cycles) that may coexist in the task pool at any time. 
%
%Prerequisites that reflect same-cycle dependencies, which is the usual
%case, should mention the host task's own cycle point, expressed as
%\lstinline=$(CYCLE_TIME)= in task definition files.
%
%For intercycle dependencies, the cycle point in a prerequisite message
%should be expressed as some offset from the task's own cycle point, e.g.\
%\lstinline=$(CYCLE_TIME - 6)=. However, the only intercycle dependencies
%you are likely to encounter (see the TopNet model in EcoConnect,
%Section~\ref{EcoConnect}, for a counter example) are the restart
%dependencies of your warm cycled forecast models, and the prerequisites
%and outputs for these are now registered automatically by cylc.
%
%\subsubsection{Message Form}
%
%The exact form of the messages does not matter so long as the
%prerequisites match their corresponding and outputs. For example, if
%the message, 
%\begin{lstlisting}
%"storm surge forecast fields ready for $(CYCLE_TIME)"
%\end{lstlisting} 
%is registered as an output by the task that generates said forecast
%fields, then the exact same message should be registered as a
%prerequisite by any task that requires that data as input
%(presumably storm surge postprocessing tasks in this case). 
%
%\subsubsection{Message Content}
%
%Prerequisites and outputs typically refer to the completion of a file or
%a group of files, but it can be any event that a task could conceivably
%trigger off: database interactions, download of data from a network,
%copying or archiving of files, etc.
%
%For single file outputs the cylc message could include the actual
%filename:
%\begin{lstlisting}
%"file surface-pressure-$(CYCLE_TIME).nc ready for use"
%\end{lstlisting}
%but there is no need to do this (see {\em Message Truth} below); you
%might as well adopt a message format that applies equally well to
%more general events and multi-file outputs:
%\begin{lstlisting}
%"surface pressure fields ready for $(CYCLE_TIME)"
%\end{lstlisting}
%
%
%\subsubsection{Message Truth}
%
%{\em Cylc does not check that incoming messages are true.}  For example,
%if the message refers to completion of a particular output file, cylc
%does not check that the file actually exists as the reporting task
%claims it does. There are two reasons for this: (1) cylc does not place
%any restriction on the kind of event that can be used as a task trigger,
%so it would be next to impossible for it to verify outputs in general,
%and (2) there is actually no need for cylc to check because the tasks
%themselves must necessarily do it, and they must immediately report
%problems back to cylc before aborting (or in the worst case, neglect to
%check and then fail for lack of required inputs, with the same result).
%
%
%\subsubsection{Uniqueness}
%
%Prerequisites need not be unique; i.e.\ multiple tasks can trigger off
%the same event.
%
%Outputs should probably be unique; otherwise a task that depends on a
%particular output will trigger off the first task to provide it.
%

   
% automatic post-intervention recovery from nasty 
%        failures, because cylc will know about the actual restart
%        dependencies of your real tasks. For example, in the userguide
%        example suite, if the weather model (task A) fails requiring a
%        cold start 12 hours later, insert the cold start task into the
%        suite (at failure time + 12) and purge all downstream dependants of 
%        the failed task through to the cold start cycle. Then, tasks
%        B and C will carry on as normal because their restart
%        prerequisites will be satisfied automatically by their
%        predecessors from several cycles ago, before the gap caused by
%        the failure.


