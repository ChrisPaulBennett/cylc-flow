\section{Site.rc Reference}
\label{SiteRCReference}

\lstset{language=bash}

*** NOT USED YET - TBD ***

\subsection[suite run directory]{suite run directory}

Cylc writes the following files to a special ``run'' directory:

\begin{myitemize}
    \item suite event log, and stdout and stderr logs, in \lstinline=SUITE/log/suite/=
    \item task stdout and stderr logs, in \lstinline=SUITE/log/job/= 
    \item suite state dump files used for restarts, in \lstinline=SUITE/state/=)
\end{myitemize}
Where \lstinline=SUITE= is the suite name.

\begin{myitemize}
    \item {\em type:} string (directory path, may contain environment variables)
\end{myitemize}

\subsection[state dump rolling archive length]{[cylc] $\rightarrow$ state dump rolling archive length}

This is the length, in number of changes, of the automatic rolling
archive of state dump files that allows you to restart a suite from a
previous state.  Every time a task changes state cylc updates the state
dump and rolls previous states back one on the archive. You'll probably
only ever need the latest (most recent) state dump, which is
automatically used in a restart, but any previous state still in the
archive can be used.  Additionally, special labeled state dumps are
written prior to actioning any suite intervention, and their filenames
are logged by cylc.

\begin{myitemize}
    \item {\em type:} integer ($\geq 1$)
    \item {\em default:} $10$
\end{myitemize}


\subsection{[logging]}

\subsubsection[roll over at start-up]{[suite logging] $\rightarrow$ roll over at start-up}

Suite logs roll over (start anew) automatically when they reach the
configured maximum size, and whenever the suite is started or restarted.

\begin{myitemize}
    \item {\em type:} boolean
    \item {\em default:} True
\end{myitemize}


