
Task proxy objects are uniquely identified by task name and cycle time,
which are exported by cylc into the task execution environment as
\lstinline=$TASK_NAME= and \lstinline=$CYCLE_TIME=. This messaging
interface then automatically knows which task is invoking it (i.e. the
task itself does not need to know the name by which cylc refers to it,
although it too can get the name from the environment if it wants to).

For manual use, note that the messaging interface needs the Pyro
nameserver group name where other cylc commands can infer the group name
from the registered system name. This is because the calling task might
be not be executing under the same username as the scheduler. For
non-manual use cylc exports \lstinline=$CYLC_NS_GROUP= into the task
execution environment for this purpose. 

Message command usage is not consistent with other commands, which
take a system name as the final argument, because in normal usage 
the message is important thing, and the running tasks using this
command know (via the environment in which they execute) what system
they belong to.
