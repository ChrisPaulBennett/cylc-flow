Registration is only required if you want to start scheduling a system
(\lstinline=cylc start= or \lstinline=restart=), run a single task
from a system (\lstinline=cylc submit=), or access system-specific
descriptive information (\lstinline=cylc info=), because these
commands need to access system-specific code modules.

Other cylc commands interact with a running target system via the Pyro
Nameserver; for these you need to know the registered name of the
system, and the username under which it is running (use 
\lstinline=cylc ping= to discover which systems are running), but you
don't need to have registered it yourself. 

The ability to run several instances of a single system at once could
allow, for example, running several historical case studies over
different time periods at once, or running a case study at the same time
as the real time operation, using the exact operational system (which
guarantees the two systems are identical). The alternative to this would
be to (i) copy the system definition directory to create an entirely new
system based on the original, and (ii) make sure the task definitions
also invoke different scripts/programs to run the real tasks.

Note that the cylc lockserver prevents multiple instances of a system
from running at the same time, even under different names or different
users, unless the system config file says the system can handle this.
