Registration serves two purposes: (1) cylc commands that need to access
system-specific modules can do so without the user having to type the
full system definition directory path on the command line; and (2) the
registered name is used, along with your username, to distinguish
between different systems using the Pyro nameserver at the same time. 

Registration is only required if you want to start scheduling a system
(\lstinline=cylc start=), run a single task from a system 
(\lstinline=cylc run-task=), or print system-specific descriptive
information (\lstinline=system-info=), because these commands need to
access system-specific code modules.

Other cylc commands interact with a running target system via the Pyro
nameserver; for these you need to know the registered name of the
system, and the username under which it is running (see 
\lstinline=cylc list-all=), but you don't need to have registered it
yourself. 

The registered name under which a system is running is made available in
the task execution environment as \lstinline=$SYSTEM_NAME=.  If system
tasks are configured to use the registered name in all important input
and output directory paths, then multiple-registration allows you to run
multiple real or dummy mode instances of that system at the same time
without any interference between them. The userguide example system can
do this. It also means that multiple users can run the same installed
version of the example system at the same time (in fact different users
can register a system under the same name because username is also used
in constructing the Pyro nameserver groupname that distinguishes
different running systems). It may also be useful, at least in
relatively small systems that are not using all available computational
resource, to run several historical case studies at once, over different
time periods, using the same system; or perhaps to run a case study at
the same time as the real time operation using the actual operational
system - an easy way to guarantee that two systems really are identical.
Of course the alternative to this is to copy the system definition
directory and thereby create an entirely new system based on the
original.

