The task processing loop is where task proxy objects negotiate
dependencies, tasks that are ready to run are submitted, and so on. To
avoid unnecessary processing cylc is designed to enter the loop only
when a task proxy changes state (as a result of an incoming message from
a running task, for instance).  If a bug in cylc resulted in this failing
to happen when it should at some point in system evolution - i.e.\ one
or more tasks have changed state but the system does not enter the loop
to determine the consequences of that change - the system may appear to
be ``stuck'' and a quick nudge will get it going again.  {\em This should
never happen!}

See also {\em Diagnosing A Stalled System},
Section~\ref{DiagnosingAStalledSystem}.


