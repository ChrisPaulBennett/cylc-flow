
See also {\em Task Wrapping} (Section~\ref{TaskWrapping}), which allows
cylc to control unmodified external tasks.

\lstset{language=bash}

The cylc task messaging commands read the task name and cycle time from
their execution environment and use these to target the correct task
proxy object in the running scheduler.

If a task is executed manually or by \lstinline=cylc submit=, the cylc
messaging interface just prints messages to stdout or stderr.

%  o If 'cylc task-started' cannot acquire a task lock it will report
%    failure to cylc and exit with error status. This implies another
%    instance of the same task (NAME%CYCLE) could still be running,
%    perhaps left over from a recent hard shutdown of the system. If so
%    we must exit manually to avoid the ERR trap - which would otherwise
%    call task-failed a second time and erroneously remove the lock.

%# SUB-PROCESSES THAT DO NOT DETACH FROM THIS SHELL:
%#  o MUST EXIT WITH NON-ZERO STATUS ON FAILURE
%#    + so this script can check for success
%#  o MUST NOT CALL cylc task-started OR -finished
%#    + this script does that
%#  o MAY CALL cylc task-message AND -failed (see "cylc-aware", below)

%# IF THIS SCRIPT INVOKES SUB-PROCESSES THAT DETACH FROM THIS SHELL 
%# (e.g. this script submits a job to loadleveller and then exits)
%# the "task" is not completed by this script so the sub-process:
%#  o MUST CALL cylc task-finished OR task-failed when it isdone.
%
%# CYLC-AWARE SUB-PROCESSES (which use 'cylc task-failed' on error)
%#  o messages are logged by cylc, including the reason for a failure
%#  o failure must be detected inline in this script or the ERR trap
%#    will result in a second call to 'cylc task-failed'.
