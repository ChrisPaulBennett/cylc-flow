
See also cylc task wrapping documentation, Section~\ref{TaskWrapping},
which allows cylc to control unmodified external tasks (at the cost that
internal outputs are only reported when the task finishes).

\lstset{language=bash}

The following variables are automatically exported by cylc into
the execution environment of a each task:
\begin{itemize}
   \item \lstinline=$TASK_NAME= - what cylc calls the task (not the
       filename of the task script). 
   \item \lstinline=$CYCLE_TIME= - task-specific forecast cycle time.
   \item \lstinline=$CYLC_NS_GROUP= - Pyro nameserver group name used by the system.
   \item \lstinline=$CYLC_NS_HOST= - Pyro nameserver hostname.
\end{itemize}

When called by a task script, \lstinline=cylc message= also has access to
these variables, which enables it to communicate with the correct task
proxy object within the running scheduler. 

The only one of these variables that you should need to use explicitly
within task scripts is the cycle time, \lstinline=$CYCLE_TIME=.

Tasks may also have access to custom task-specific environment variables
set in the task definition file.

If a task script is executed manually or by \lstinline=cylc run-task=
then \lstinline=cylc message= will print to stdout instead of attempting
to communicate with a task proxy object registered in a Pyro nameserver.
