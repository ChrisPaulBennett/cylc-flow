
See also cylc task wrapping documentation, Section~\ref{TaskWrapping},
which allows cylc to control unmodified external tasks (at the cost that
internal outputs are only reported when the task finishes).

\lstset{language=bash}

The following variables are automatically exported by cylc into
the execution environment of a each task:
\begin{itemize}
   \item \lstinline=$SYSTEM_NAME= - registered name of the running system.
   \item \lstinline=$TASK_NAME= - how cylc refers to the task (not the
       filename of the task script). 
   \item \lstinline=$CYCLE_TIME= - task-specific forecast cycle time.
   \item \lstinline=$CYLC_NS_GROUP= - Pyro nameserver group name used by the system.
   \item \lstinline=$CYLC_NS_HOST= - Pyro nameserver hostname.
\end{itemize}

\lstset{language=bash}
When called by a task script, \lstinline=cylc message= read the task
name and cycle time from the environment and uses these to target the
correct task proxy object within the running scheduler.
\lstset{language=cylctaskdef=}
A task can also access any task-specific
environment variables set in the task definition file
\lstinline=ENVIRONMENT= key.

\lstset{language=bash}

If a task script is executed manually or by \lstinline=cylc run-task=
then \lstinline=cylc message= will print to stdout instead of attempting
to communicate with a task proxy object registered in a Pyro nameserver.
