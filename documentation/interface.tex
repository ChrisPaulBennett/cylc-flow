%\documentclass[11pt,a4paper]{report}
\documentclass[11pt,a4paper]{article}

\usepackage{listings}
\usepackage{amsmath}

\lstset{language=Python}

\title{Connecting Sequenz To Real Tasks}
\author{Hilary Oliver, NIWA}

\begin{document}

\maketitle
\tableofcontents

\section{Overview}

Divide your system into a set of distinct tasks. A {\em task} is a group
of processes to be treated as a unit for scheduling purposes and
invoked via a single {\em task control script}. Then:

\begin{itemize}

    \item \textbf{Write a sequenz task definition file for each task}.
    This specifies task properties such as name, valid run times,
    prerequisites and postrequisites, and the task control script used
    to invoke the task. Task definition files are used in
    auto-generating Python class code to define the task objects managed
    by sequenz.
    
    \item \textbf{Modify the external task control scripts}. Completion
    of postrequisites, etc., must be communicated back to sequenz. Note,
    however, that simple task-completion based metascheduling of
    existing tasks can be done via simple a simple wrapper that reports
    task startup, invokes the task, and reports task completion, without
    modifying existing external tasks in any way.

\end{itemize}

\section{Task Definition Files}

A {\em Task Definition File} defines the properties of a sequenz task:
name, valid run times, prerequisites and postrequisites, the task
control script used to invoke the task, etc.  These files are parsed by
a sequenz utility program that auto-generates Python to define the
sequenz task classes:

\lstset{language=sh, numbers=left}

{\small
\noindent
\rule{5cm}{.2mm}
\begin{lstlisting}
# generate class code for tasks defined in
# the ecoconnect-operations sub-directory:

cd task-definitions
generate_task_classes.py ecoconnect-operations
\end{lstlisting}
}

\subsection{Simple Example}

\lstset{language=sh, numbers=left}

{\tiny
\noindent
\rule{5cm}{.2mm}
%\begin{lstlisting}
\lstinputlisting{../task-definitions/simple_example.def}
%\end{lstlisting}
}

\subsection{Full Task Definition Template}

\lstset{language=sh, numbers=left}

{\tiny
\noindent
\rule{5cm}{.2mm}
%\begin{lstlisting}
\lstinputlisting{../task-definitions/full_template.def}
%\end{lstlisting}
}

\subsection{More Complex Task Behaviour}

\section{Task Control Script Requirements}

Each external task control script must:

\begin{itemize}
\item report (to sequenz) when the has task started
\item report when each registered task postrequisite has completed
\item report when the task has finished
\end{itemize}

In addition, arbitrary non-postrequisite messages can be sent for
debugging, logging, or progress monitoring purposes.

Note that postrequisite completion messages could potentially be 
sent by lower level scripts that are invoked as the task runs. 

\subsection{Task Messaging Interface}

A sequenz utility script \verb=task_message= communicates with sequenz:


\subsection{Simple Example}

\section{How Sequenz Launches External Tasks}


\end{document}
