%\documentclass[11pt,a4paper]{report}
\documentclass[11pt,a4paper]{article}

\usepackage{listings}
\usepackage{amsmath}

\lstset{language=Python}

\title{Connecting Sequenz To Real Tasks}
\author{Hilary Oliver, NIWA}

\begin{document}

\maketitle
\tableofcontents

\section{Overview}

Divide your system into a set of distinct tasks. A {\em task} is a group
of processes to be treated as a unit for scheduling purposes and
invoked via a single {\em task control script}. 

\subsection{Define Each Task}

A \textbf{task definition file} must be written for each task. These
files specify task properties such as name, valid run times,
prerequisites and postrequisites, and how to invoke the real external
task.
    
\subsection{Task Messaging}

The external tasks (or rather the scripts that run them) must be 
modified so that completion of postrequisites is reported back to
sequenz. 

Note however that simple task-completion based metascheduling of
existing tasks can be done via simple a simple wrapper that reports task
startup, invokes the task, and reports task completion, without
modifying existing external tasks in any way.


\section{Task Definition Files}

A {\em Task Definition File} defines the properties of a sequenz task:
name, valid run times, prerequisites and postrequisites, the task
control script used to invoke the task, etc.  A sequenz utility, 
\verb=generate_task_classes.py= parses task definition files and 
auto-generates Python code that defines the objects that represent
each task within sequenz.

\subsection{Typical Example}

Most tasks have only cotemporal (same reference time) upstream and
downstream dependencies (recall that every task is implicitly dependent
on the previous instance of its own class, in sequenz). The following
task definition file is sufficient to completely define such a task.
Different postrequisites can be specified depending on reference time. 

\lstset{language=sh, numbers=left}

{\tiny
\noindent
\rule{5cm}{.2mm}
%\begin{lstlisting}
\lstinputlisting{../task-definitions/typical.def}
%\end{lstlisting}
}

\subsection{Full Task Definition Template}

\lstset{language=sh, numbers=left}

{\tiny
\noindent
\rule{5cm}{.2mm}
%\begin{lstlisting}
\lstinputlisting{../task-definitions/full_template.def}
%\end{lstlisting}
}

\subsection{More Complex Task Behaviour}

\textit{In EcoConnect: streamflow and topnet}

For tasks that have non-cotemporal upstream dependencies and/or need to  
override task class methods to define new behaviour, you have to 
write the Python class source directly. 

\section{Task Messaging Mechanism}

Each external task must:

\begin{itemize}
\item report (to sequenz) when the task has started
\item report when the task has finished
\item report when every other registered task postrequisite has
completed
\end{itemize}

(Technically, the `started' and `finished' messages are just
postrequisites too, but they are special in that every task
must have them).

In addition, tasks can optionally:

\begin{itemize}
\item report any arbitrary unregistered (i.e. non-postrequisite)
messages, for debugging, logging, or progress monitoring purposes.
\end{itemize}

All incoming messages are logged by sequenz, but only postrequisite
messages affect the state of their target task objects.

Task messages don't necessarily have to originate from top level task
control scripts. It's a probably a good idea to do this if possible, but
lower level scripts that are invoked as the task runs can communicate
directly with sequenz if necessary.

\subsection{Task Messaging Interface}

A sequenz utility script \verb=task_message= communicates with sequenz:


\subsection{Simple Example}

\section{How Sequenz Launches External Tasks}


\end{document}
