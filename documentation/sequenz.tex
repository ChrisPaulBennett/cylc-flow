%\documentclass[11pt,a4paper]{report}
\documentclass[11pt,a4paper]{article}

\usepackage{listings}
\usepackage{amsmath}

\lstset{language=Python}

\title{Dynamic Sequencing of Interdependent Tasks, with
Application to Chained Environmental Forecasting Models}

\author{Hilary Oliver, NIWA}

\begin{document}

\maketitle
\tableofcontents


\section{Dynamic Sequencing}

\subsection{Overview}

{\em Dynamic Sequencing} is an extremely flexible way to control the
execution order of a pool of interdependent tasks\footnote{A {\em task}
is any group of processes treated as a single entity for scheduling
purposes.} by enabling them to resolve their own internal dependencies
through interaction, so that {\em correct sequencing emerges naturally
at run time}. In other words, it determines when each task is ready to
run, in the presence of prerequisites that must be satisfied by other
tasks in the system, without use of explicit sequencing logic. Once a
task is ready to run it is submitted to a batch queue scheduler. Tasks
do not all have to run on the same machine.  In general, task
dependencies result in multiple parallel streams of execution that can
branch (when one task generates output for several downstream task) and
merge (when one task takes input from several upstream task), which
presents a difficult control system design problem.  


\subsection{Sequenz}

{\em Sequenz} is a Python program that implements the dynamic sequencing
concept. It is used to run NIWA's EcoConnect operational environmental
forecasting system, which consists of global weather and sea state
models that drive regional weather, sea state, storm surge, tide, and
river flow models, and many associated pre- and post-processing tasks.
The tasks run on heterogenous distributed hardware, on an operational
schedule that repeats several times daily (between one and twenty four
times, depending on the task).  It automatically runs tasks from many
many different forecast cycles simultaneously, where dependencies allow,
for maximum throughput\footnote{This is useful when catching up to real
time after system delays, and for running the entire system over
historical periods for which all the initial input data is already
available.} ({\em multiflight} operation). Tasks can be switched on and
off trivially, and new ones can be added by simply defining their
prerequisites (without reference to who will satisfy them) and outputs
(without reference to who will use them). The system can be restarted in
arbitrarily complex states after maintenance, and sequencing of all
system tasks can be fully tested in an accelerated-time dummy mode
without actually running the real external tasks. 

Sequenz relies on Object Oriented Programming techniques to construct
self-contained {\em task objects} that represent each external system
task.  A task object knows its own prerequisites and outputs, keeps its
internal state in sync with the external task via remote method
calls\footnote{Remote method calls are achieved using the Python Remote
Object ({\em Pyro}) protocol.}, can interact with other task objects to
find out if any of its prerequisites have been satisfied by their
completed outputs, and can launch the external task when the time comes.
Tasks do not need to know anything about the other tasks in the system,
just their own prerequisites and outputs as would be required for them
to run in isolation. They do not even need to know who is supposed to
satisfy their prerequisites because they can ask {\em all} other tasks.
The total absence of explicit sequencing logic makes this method
extremely flexible: tasks can be switched on and off, and new ones
added, without changing the main program at all. Each task runs at the
right time so long as the system manager ensures that the task pool is
consistent (i.e. that each task's prerequisites will be satsified by
another task's outputs).


\subsection{Applicability}

The dynamic sequencing concept is very general and could in principle be
implemented for any set of interdependent tasks. Sequenz, however, is
somewhat specialized toward systems, such as EcoConnect, that normally
need to keep cycling indefinitely. In particular, each task has an
associated {\em reference time}, representing the nominal start time of
its associated forecast run, and all prerequisites and outputs are
assumed to depend on it in some way.  One-off task pools with no
particular time dependence, however, are much simpler than this, and it
would not be difficult to strip all reference time handling from the
program.

In addition, EcoConnect operates in a well defined environment so that,
for example, each task knows exactly what its input files look like
(filenaming conventions) and where to get them from (e.g. {\em task X}'s
output directory). Consequently, for file-based prerequisites, the
controller does not need to know the file's actual location or check for
its existence, because the external tasks already do that. It could,
however, easily be made to pass file locations between tasks so that all
input/output filenames and paths could be defined centrally in the
control program itself.  


\subsection{Alternatives}

There are currently no other systems in existence that can handle
general {\em multiflight} sequencing, as far as the author is aware.
This section just contains a brief discussion of some alternatives that
were considered prior to beginning development of sequenz.

\subsubsection{Explicit Sequencing}

The obvious algorithm to try first is a {\em Finite State Machine} that
uses explicit sequencing logic to enforce a predetermined order of
events: {\em if task A has finished and task B has generated file X,
then run task C}, and so on. This is relatively easy to understand in
simple systems, but the sequencing logic becomes increasingly convoluted
as system complexity increases, and with any attempt to build in
flexibility (e.g. by allowing tasks to be switched on and off easily,
which effectively changes the predetermined order). In addition, for
forecast systems (and similar) it does not seem possible to generalise
this to multiflight operation without prohibitively complex
coding.\footnote{Note that a sequence of overlapping independent
single-flight controllers does not work, in general, because of
intercycle dependencies: $foo(T\negmedspace +\negmedspace 1)$ may depend
on $foo(T)$, for example, because forecast models usually start from a
``background state'' generated by the previous forecast.} 


\subsubsection{Dependent Batch Schedulers}

Some batch queue schedulers now allow simple job dependencies ({\em
don't run Y until X is finished}) but are still far too rudimentary for
our purposes. 


\subsubsection{Generic Control Frameworks}

Use of a generic control framework of some kind was also
considered, but nothing that was obviously well suited came to light. In
any case, in the author's opinion, the dynamic sequencing design
described in this document simplifies the problem to such an extent that
it is unlikely any generic framework could compete.  


\section{Sequenz Implementation}

To implement dynamic sequencing we need a specific {\em task management}
scheme that controls when tasks will be created and destroyed, and so
on. There are almost too many options here; for instance: should tasks
be created all at once, in cotemporal batches, or one at a time as their
predecessors finish? The consequences of choices such as this have to
be evaluated carefully, which isn't easy because of the inherent
complexity of multiflight operation. At the simple end of the task
management spectrum, for example, we could create a whole month's worth
of tasks at once and just let them interact until everything runs to
completion. Nothing would run out of sequence, {\em but} system
monitoring would be difficult because of the sheer number of tasks
involved, the end of the month would present an artificial barrier to
multiflight operation, tasks that lack prerequisites would all want to
run at once, and system restarts would be overly complicated. 

\subsection{Main Algorithm}

The algorithm below operates on a single pool of interacting task
objects, has simple start up and continuous operation, and is relatively
easy to monitor (task objects exist by the time they are needed but not
for too long before that, and not for too long after they are finished). 

The simplicity of the dynamic sequencing algorithm is clear from the
following code listing, taken directly from the main program:

{\small
\noindent
\rule{5cm}{.2mm}
\begin{lstlisting}
# (startup initialization code omitted)

while True: # MAIN LOOP

   if task_base.state_changed:
       # PROCESS ALL TASKS whenever one has changed state
       # as a result of a remote task message coming in: 
       # interact, run, create new, and kill spent tasks
       #---
       task_pool.process_tasks()
       task_pool.dump_state()
       if task_pool.all_finished():
           clean_shutdown( "ALL TASKS FINISHED" )

    # REMOTE METHOD HANDLING; handleRequests() returns 
    # after one or more remote method invocations are 
    # processed (these are not just task messages, hence 
    # the use of task_base.state_changed above).
    #---
    task_base.state_changed = False
    pyro_daemon.handleRequests( timeout = None )

# END MAIN LOOP
\end{lstlisting}
}

\subsection{Details}

\subsubsection{Startup and Initialization}

An initial reference time and list of task object names are read in from
the config file, then each task object is created at the initial
reference time {\em or} at the first subsequent reference time that is
valid for the task type. Optionally, we can tell the controller to
reload the current state dump file (which may have been edited); this
will override the configured start time and task list. After startup,
new tasks are created only by {\em abdication} (below).

An initial run through the {\em task processing} code, by virtue of the
fact that the main loop starts with task processing, causes tasks with
no prerequisites (e.g. {\em downloader}) to enter the {\em running}
state and launch their external tasks immediately. Otherwise ({\em or}
if there are no tasks that lack prerequisites) nothing will happen.



\subsubsection{Creation of New Tasks}

New tasks are created by abdication, i.e. create $foo(T\negmedspace
+\negmedspace 1)$ if $foo(T)$ is {\em finished}.  Task abdication
ensures that $foo(T\negmedspace +\negmedspace 1)$ won't run before
$foo(T)$ finishes, without imposing explicit intercycle prerequisites
that would require special treatment at startup (when there is no
previous cycle).  It also ensures that tasks with no prerequisites, e.g.
{\em downloader} and {\em nztide}, won't all try to run at once.
Tasks are not deleted immediately on abdication (see below).


\subsubsection{Task Interaction} 

Each task keeps track of which of its postrequisites are completed, and
asks the other tasks if they can satisfy any of its prerequisites.  The
fact that task objects do not need to know who is supposed to satisfy
their prerequisites (because they can ask, indiscriminately, every other
task in the system) makes the task interaction (sequencing!) algorithm
almost trivial. The fact that most of these interactions are fruitless
is of no consequence. 

{\small
\noindent
\rule{5cm}{.2mm}
\begin{lstlisting}
class task_pool( Pyro.core.ObjBase ):
    # ...
    def interact( self ):
        # get each task to ask all the others if 
        # they can satisfy its prerequisites
        #--
        for task in self.tasks:
            task.get_satisfaction( self.tasks )
    # ...
\end{lstlisting}
}

\subsubsection{Running Tasks}

Each task object can launch its associated external task, and enter the
{\em running} state if its prerequisites are all satisfied, any existing
older tasks of the same type are already {\em finished}, and fewer than
{\em MAX\_ RUNAHEAD} finished tasks of the same type still exist (this
stops tasks with no prerequisites from running ahead indefinitely).

\subsubsection{Pyro Remote Method Calls}

The Pyro request handling loop executes remote method calls coming in
from external tasks, and returns after at least one call was handled.
Pyro must be run in non-default single-threaded mode (see Appendix
\ref{pyro-appendix}).

\subsubsection{Dumping State} 

The current state (waiting, running, or finished) of each task is
written out to the {\em state dump file}.  This provides a running
snapshot of the system as it runs, and just prior to shutdown or
failure. The controller can optionally start up by loading the state
dump (which can be edited first). Any 'running' tasks are reloaded in
the 'waiting' state.

\subsubsection{Removing Spent Tasks} 

A task is spent if it finished {\em and} no longer needed to satisfy the
prequisites of any other task. Most tasks are only needed by other
cotemporal downstream tasks; these can be removed when they are finished
{\em and} older than the oldest non-finished task. For rare cases that
are needed by tasks in later reference times (e.g. nzlam post
processing: multiple hourly topnet tasks need the same most recent
previously finished 06 or 18Z nzlam post processing task), each
non-finished task reports its {\em cutoff reference time} which is the
oldest reference time that may contain tasks still needed to satisfy its
own prerequisites (if it is waiting) or those of its immediate
post-abdication successor (if it is running already), then the task
manager can then kill any finished tasks that are also older than the
oldest task cutoff time.

\subsubsection{Removing Lame Tasks} 

Tasks that will never run (because their prerequisites cannot be
satisfied by any other task in the system) are removed from the {\em
oldest batch} of tasks.  If not removed they would prevent the spent
task deletion algorithm from working properly. Lame tasks can only be
detected in the oldest task batch; in younger batches some tasks may yet
appear as their predecessors abdicate.

Lame tasks are abdicated rather than just deleted, because their
descendents will not necessarily be lame: e.g. if the system is started
at 12Z with topnet turned on, all topnet tasks from 12Z through 17Z will
be valid but lame, because they will want to take input from a
non-existent nzlam\_post from 06Z prior to startup. However, the
presence of lame tasks may indicate user error: e.g. if you forget
to turn on task type $foo$ that supplies input to task type $bar$,
any instance of $bar$ will be lame.


\section{Dummy Mode}

Dummy mode allows complete testing of the control system without running
any real external tasks\footnote{The only difference between dummy mode
and real operation, as far as the controller is concerned, is that
external dummy tasks are not delayed by resource contention.}. When it
is ready to run, a task object will launch an external dummy program
that (i) gets a list of postrequisites from the parent task object and
then (ii) reports back that each one is satisfied at the (estimated)
right time relative to an accelerated dummy clock. Dummy tasks therefore
complete in approximately the same dummy clock time as the real tasks do
in real time. An initial dummy clock offset relative to the initial
reference time can also be specified, which allows simulation of the
transition between catchup and real time operation. Log messages are
stamped with dummy clock time instead of real time.

The same script is used for all external dummy tasks but it has special
behaviour in certain cases: the dummy downloader ``waits for incoming
files'' until 3:15 past its reference time, and the dummy topnet ``waits
for streamflow data'' until 0:15 past its reference time.

The dummy clock can be bumped forward a number of hours by remote
control, while the system is running. This affects the postrequisite
timing of running tasks correctly, but if it causes a running task to
finish immediately the next task in line will still start from the
beginning no matter how big the bump.


\section{Program Usage}

All user-configurable parameters are set in \verb#config.py#. There is
one commandline option to force a restart from the state dump file
(which may have been edited) instead of the configured start time and
task list:

\lstset{language=sh}

{\small

\noindent
\rule{5cm}{.2mm}
\begin{lstlisting}
ecocontroller [-r]
Options:
    + most inputs should be configured in config.py
    + [-r] restart from the state dump file
\end{lstlisting}
}

\lstset{language=Python}

\subsection{Config File}

{\small
\noindent
\rule{5cm}{.2mm}
\lstinputlisting{../config.py}
}

\appendix

\section{Essential OOP Concepts}

``Task objects'' in sequenz are {\em objects} in the Object Oriented
Programming sense, and the simplicity of the sequencing algorithm
depends critically on their {\em polymorphic} nature.  This section
contains a minimal introduction to OOP concepts needed to understand
this.  Refer to any OOP reference for more detail.

\subsection{Classes and Objects}

A {\em class definition} declares the properties of {\em objects} or
{\em instances} of that class; objects are self-contained entities that
can be assigned to a variable and have specific {\em state} (data) and
{\em behaviour} (functions or ``methods'' that operate on the data and
are the means by which objects interact with the outside world).    

For example, a $shape$ class could define a $position$ data member that
describes the location of each shape object, a $move()$ method that
causes a shape object to alter its position, and a $draw()$ method that
causes it to display itself in the right place on screen.

\subsection{Inheritance}

A {\em derived class} or {\em subclass} inherits the properties (methods
and data members) of its {\em parent class}. It can also {\em override}
specific parent class properties, or add new properties that aren't
present in the parent class. Calling a particular method on an object
invokes the object's own class method if one is defined, otherwise the
parent class is searched, and so on down to the {\em base class} at the
root of the object's inheritance tree. 

For example, we could derive a $circle$ class from $shape$, adding a
`radius' data member and overriding the $draw()$ to get circle objects
to display themselves as actual circles.  Because we didn't override the
$move()$ method, calling $circle.move()$ would invoke the base class
method, $shape.move()$. 


\subsection{Polymorphism}

Polymorphism is the ability of one {\em type} to appear as and be used
like another type.  In OOP languages with inheritance, this usually
refers to the ability to treat derived/sub-class objects as if they were
members of a parent class.  In particular, a group of mixed-type objects
can all be treated as members of a common parent class. For example, we
could construct a list of $shape$ objects from $circles$, $triangles$,
and $squares$; calling $[list member].draw()$ will invoke the right
derived class $draw()$. This is a very powerful mechanism because {\em
it allows unmodified old code to call new code}: if we later derive an
entirely new kind of shape ($hexagon$, say) with it's own unique
behaviour, the existing program, without modification, will process the
new objects in the proper hexagon-specific way.


\section{Threading in Pyro} \label{pyro-appendix}

With Pyro in {\em single threaded mode}, \verb#handleRequests()# returns
after {\em either} a timeout has occurred {\em or} at least one request
(i.e.  remote method call) was handled. With \verb#timeout = None# this
allows us to process tasks {\em only} after remote method invocations
come in.  Further, we can detect the remote calls that actually change
task states, and thereby drop into the task processing code only when
necessary, which minimizes non-useful output from the task processing
loop (e.g. in dummy mode there are a lot of remote calls on the dummy
clock object, which does not alter tasks at all). 

In {\em multithreaded mode}, \verb#handleRequests()# returns immediately
after creating a new request handling thread for a single remote object
and thereafter remote method calls on that object come in asynchronously
in the dedicated thread. This is not good for the dynamic sequencing
algorithm because tasks are only set running in the task processing
block which can be delayed while \verb#handleRequests()# blocks waiting
for a new connection to be established even as messages that warrent
task processing are coming in on existing connections. The only way
around this is to do task processing on \verb#handleRequests()# timeouts
which results in a lot of unnecessary task processing when nothing
important is happening.


\section{YET TO BE DOCUMENTED}

\begin{itemize}
 \item usage
 \item logging
 \item debugging
 \item adding new task definitions
 \item system monitors
 \item remote control: 
    \begin{itemize}
    \item clean shutdown of pyro
    \item bumping the dummy clock forward
    \end{itemize}
 \item external task messaging interface script
 \item dummying out tasks in real mode
 \item how to set up partial parallel test systems
\end{itemize}

\section{Miscellaneous Notes}

(To be incorporated into the main documentation, or deleted).

\subsection{catchup mode}

Note that ``correct model sequencing'' is not equivalent to ``orderly
generation of products by reference time''.  Nzlam can run continuously,
regardless of the downstream processing that depends on it.

Catchup mode is now task-dependent, rather than a property of the whole
system (as it is if only tasks from the same reference time can run at
once).  Where this matters, it can be detected by the relevant external
task. E.g. if the external topnet(T) task starts up at real time t
greater than the streamflow data time for T (i.e. $T+15$ min), i.e. the
required streamflow data is already available, then we're still in
catchup. If, on the other hand, the topnet task finds that it has to
wait for its streamflow data time to arrive, then we're caught up to
real time.  This matters because topnet is allowed to run ahead by a
different amount of time depending on whether we're in catchup mode or
not.


\subsection{controlling when a task executes}

\begin{itemize}
 \item  prerequisites
 \item artificial prerequisites (e.g. make nztide depend on nzlam)
 \item delayed instantiation (a task can't run if it doesn't exist yet).
 \item other contraints based on, for example, the number of previous
 instances that still exist in the system.
\end{itemize}


\subsection{fuzzy prequisites}

{\em Exact prerequisites} (most tasks): times are specified exactly,
relative to the task's own reference time.  E.g. {\em file foo\_{T}.nc
ready} where T is the task's reference time.

{\em Fuzzy prerequisites} (topnet): a time boundary is specified
relative to the task's own reference time; any task with a reference
time greater than or equal to the boundary time can satisify the
prerequisite.

\subsection{Task Messaging}

To the task objects, outputs are just {\em messages} indicating that the
external task has reached a significant waypoint, such as generation of
a particular output file.

\section{To Do}

\begin{itemize}

\item
 retrofit proper exception handling throughout (currently used sparsely and 
  inconsistently)

\item
 potential bug in the free task class: will never run if too many
 previous instances exist that are newer than the task deletion cutoff

\item
 remote logging from messaging and dummy task scripts, in case the
 controller itself is dead.

\end{itemize}

\end{document}
