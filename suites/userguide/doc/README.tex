
%THIS FILE IS INCLUDED AUTOMATICALLY IN THE USERGUIDE LaTeX SOURCE
%DURING DOCUMENT PROCESSING

This suite can run in real or dummy mode; the resulting job schedules
should be the same apart from small timing differences.  The task
scripts invoked in real mode create zero-sized output files with the
\lstinline=touch= command, and input files are ``read'' by simply
detecting the file's existence.  All tasks read and write files from a
common location, for the sake of simplicity. 

See {\em Handling Real Dependencies}
(Section~\ref{HandlingRealDependencies}) for how to manage input and
output files in a more realistic suite.  The Unix 'sleep' command is
used to get the example suite tasks to execute quickly, via the 
\lstinline=$TASk_RUN_TIME_SECONDS= variable defined in the suite config
file.  The suite contains the following tasks:

\begin{itemize}
    \item {\bf A} - an atmospheric model, it depends on external real
        time obs and its own restart file.
    \item {\bf B} - a sea state model, it depends on outputs generated
        by A, and its own restart file.
    \item {\bf C} - a storm surge model, it depends on outputs generated
        by A, and its own restart file.
    \item {\bf D} - post processes outputs generated by B and C.
    \item {\bf E} - post processes outputs generated by B.
    \item {\bf F} - post processes outputs generated by C.
    \item {\bf X} - a {\em contact task} that gets the obs data required
    by A, at 1 hour past its cycle time. X has no prerequisites and
    therefore runs immediately when its contact time has passed.
\end{itemize}

There are also two additional tasks not mentioned in Section~\ref{HowCylcWorks}:

\begin{itemize}
    \item {\bf startup} - a oneoff task that cleans out the suite
    workspace at startup.
    \item {\bf coldstart} - a oneoff task to provide initial restart
        inputs to the forecast models. 
\end{itemize}

Task F illustrates the use of cylc's {\em task wrapping} mechanism
(Section~\ref{TaskWrapping}) for running unmodified external tasks. 
The taskdef ENVIRONMENT key is used to put the cycle time
into the task execution environment in the form expected by the wrapped
external task (\lstinline=\$ANALYSIS_TIME= in this case). 

To run the suite follow the {\em Quick Start Guide}
(Section~\ref{QuickStartGuide}) from the `configure' step onward. 
{\em Be careful to set an initial cycle time in the past, in real mode,
or nothing will happen until until the start time is reached!}.

{\em RUNNING MULTIPLE SUITE INSTANCES AT ONCE:} All task I/O is done
under \lstinline=\$CYLC_TMPDIR=, defined at run time to include the
registered suite name (see suites/userguide/suite.config).  Thus if
you register the suite under multiple names you can run suite instances
at once without any interference between them.          

{\em GETTING TASKS TO FAIL ON DEMAND IN REAL MODE OPERATION:} For this
example suite only, before startup set the environment variable
\lstinline=FAIL_TASK= to the ID of the task you want to fail, in the
suite config file or at the command line - the specified task will
abort via \lstinline=suites/userguide/scripts/check-env.sh=.  If the
task is subsequently reset, or the suite restarted, the task will not
abort again, as if the problem had been fixed. NOTE in DUMMY MODE you
can do the same for ANY suite, with the \lstinline=--fail= commandline
option.

{\em REAL TIME OPERATION:} The tasks in this suite are designed to run
quickly (a few seconds) but if the suite catches up to real time
operation there will still be a real 6 hour delay between cycles - so
you might want to make sure the initial cycle time is signficantly in
the past. This is not an issue in dummy mode, which runs on an
accelerated clock.       

To understand what happens when the suite starts up, see {\em
Understanding Suite Evolution}
(Section~\ref{UnderstandingSuiteEvolution}).
