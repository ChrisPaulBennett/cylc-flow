
%THIS FILE IS INCLUDED AUTOMATICALLY IN THE USERGUIDE LaTeX SOURCE
%DURING DOCUMENT PROCESSING

The packaged system defined in,

\begin{lstlisting}
[CYLC_TOP_DIR]/sys/examples/userguide/
\end{lstlisting}

is a complete working implementation of the example used to illustrate
how cylc works of in Section~\ref{HowCylcWorks}, with the dependency
diagram shown in Figure~\ref{fig-dep-two}. It can be run in real mode or
dummy mode; the result (scheduling-wise) should be essentially
identical. The task scripts invoked in real mode create zero-sized
output files with the \lstinline=touch= command, and all tasks ``read''
and ``write'' files from the one location. See the discussion in
Section~\ref{????} for how to manage input and output file locations in
a more realistic forecasting system. While this is clearly not a real
forecasting system, it has exactly the same properties as a real system
as far as scheduling is concerned.  

The Unix 'sleep' command is used to get the trivial example tasks to
execute in the right amount of time (as defined in their taskdef files).
Then, to get the real tasks to execute as quickly as their dummy mode
counterparts, the sleep time is scaled by \lstinline=$REAL_TIME_ACCEL=,
an environment variable defined in the system config module.

The system contains the following tasks:

\begin{itemize}
    \item {\bf A} - this task could be an atmospheric forecast model,
    it depends on external real time obs and its own restart file.
    \item {\bf B} - this task could be a sea state model, it depends on 
    outputs generated by A, and its own restart file.
    \item {\bf C} - this task could be a storm surge model, it depends on 
    outputs generated by A, and its own restart file.
    \item {\bf D} - this task post processes outputs generated by B and C.
    \item {\bf E} - this task post processes outputs generated by B.
    \item {\bf F} - this task post processes outputs generated by C.
    \item {\bf X} - this "external contact" task gets the obs data required
    by A, at 1 hour past its reference time. X has no prerequisites and
    therefore runs immediately, once its contact time has passed.
\end{itemize}

There are also {\bf two additional tasks}:

\begin{itemize}
    \item {\bf startup} - a oneoff task that cleans out the system
    workspace (used by all tasks for their output files, in this
    example) at startup.
    \item {\bf coldstart} - a oneoff task provides the initial restart
    prerequisites required by the forecast models (once the system
    is cycling these are provided by previous forecasts). 
\end{itemize}

{\bf Task F illustrates the use of cylc's task wrapping mechanism} for
running unmodified external tasks (i.e. task that do not know about
cylc) - see Section~\ref{TaskWrapping}. The taskdef file ENVIRONMENT key
is used to export the cycle time, in the form expected by the wrapped
external task, into the task execution environment.

The \lstinline=$NEXT_CYCLE_TIME= and \lstinline=$NEXT_NEXT_CYCLE_TIME=
variables defined in the task definition files, and read by the external
``forecast model'' scripts, are just a convenience to allow the minimalist
external tasks to avoid doing their own cycle time arithmetic to compute
the validity times of their restart outputs.

{\bf To run the example system} follow the Quick Start instructions from
the 'configure' step onward (Section~\ref{QuickSystemConfiguration}).
