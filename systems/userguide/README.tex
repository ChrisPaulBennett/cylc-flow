
%THIS FILE IS INCLUDED AUTOMATICALLY IN THE USERGUIDE LaTeX SOURCE
%DURING DOCUMENT PROCESSING

This example system can be run in real and dummy modes; the result
(scheduling wise) should be essentially identical. The task scripts
invoked in real mode create zero-sized output files with the
\lstinline=touch= command, and input files are ``read'' by simply
detecting the file's existence.  All tasks read and write files from a
common location, to simplify the system for illustrative purposes. See
{\em Handling Real Dependencies}
(Section~\ref{HandlingRealDependencies}) for how to manage input and
output files in a more realistic system. 

The Unix 'sleep' command is used to get the trivial example tasks to
execute in the right amount of time (as defined in their taskdef files),
but scaled by \lstinline=$REAL_TIME_ACCEL=, an environment variable
defined in the system config module, so that the system runs quickly 
even in real mode.

The system contains the following tasks:

\begin{itemize}
    \item {\bf A} - could be an atmospheric forecast model,
    it depends on external real time obs and its own restart file.
    \item {\bf B} - could be a sea state model, it depends on 
    outputs generated by A, and its own restart file.
    \item {\bf C} - could be a storm surge model, it depends on 
    outputs generated by A, and its own restart file.
    \item {\bf D} - post processes outputs generated by B and C.
    \item {\bf E} - post processes outputs generated by B.
    \item {\bf F} - post processes outputs generated by C.
    \item {\bf X} - a {\em contact task} that gets the obs data required
    by A, at 1 hour past its reference time. X has no prerequisites and
    therefore runs immediately its contact time has passed.
\end{itemize}

There are also two additional tasks not mentioned in Section~\ref{HowCylcWorks}:

\begin{itemize}
    \item {\bf startup} - a oneoff task that cleans out the system
    workspace (used by all tasks for their output files, in this
    example) at startup.
    \item {\bf coldstart} - a oneoff task provides the initial restart
    prerequisites required by the forecast models (once the system
    is cycling these are provided by previous forecasts). 
\end{itemize}

Task F illustrates the use of cylc's {\em task wrapping} mechanism
(Section~\ref{TaskWrapping}) for running unmodified external tasks. 
The taskdef file ENVIRONMENT key is used to put the cycle time
into the task execution environment in the form expected by the wrapped
external task (\lstinline=$ANALYSIS_TIME= in this case). 

The \lstinline=$NEXT_CYCLE_TIME= and \lstinline=$NEXT_NEXT_CYCLE_TIME=
variables defined in the task definition files, and read by the external
``forecast model'' scripts, are just a convenience to allow the
minimalist external tasks to avoid doing their own cycle time arithmetic
to compute the validity times of their restart outputs.

To run the example system follow the {\em Quick Start Guide}
(Section~\ref{QuickStartGuide}) from the `configure' step onward. Be
careful to set an initial cycle time in the past, or else nothing will
happen until until the start time is reached.

To understand what happens when the system starts up (why, for instance,
a lot of {\em X}'s all go off at once) see {\em Understanding System Evolution}
(Section~\ref{UnderstandingSystemEvolution}).

